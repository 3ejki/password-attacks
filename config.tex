%Packages
\documentclass[11pt, a4paper]{article}

% Papierformat und Seitenränder
\usepackage[verbose,a4paper,top=25mm,bottom=25mm,left=30mm,right=30mm]{geometry}
\usepackage[utf8]{inputenc}
\usepackage[T1]{fontenc}

%Legt die Schriftart auf ``Latin Modern'' fest
\usepackage{lmodern}

%1,5 facher Zeilenabstand
%\usepackage[onehalfspacing]{setspace}
% Einfacher Zeilenabstand
\usepackage[singlespacing]{setspace}

%Für neue Rechtschreibung und deutsche Silbentrennung
\usepackage[ngerman]{babel}

%Ermöglicht das Einbinden von Grafiken
\usepackage{graphicx}
% Um Grafiken fließender Text
%\usepackage{wrapfig}

% Erhöht maximale Lückengröße zwischen Wörtern
%\setlength\emergencystretch{1em} 

% Für Zitate
\usepackage[round]{natbib}
\bibliographystyle{Bibliographystyle}
%\setlength{\bibinitsep}{\baselineskip}

% %Für Anhänge
% %\usepackage[titletoc,title]{appendix}

% %Nummeriert alle Ebenen von Part (-1) bis Subparagraph (5)
% \setcounter{secnumdepth}{3}
% %Nimmt alle Ebenen von Part (-1) bis Subparagraph (5) ins Inhaltsverzeichnis auf
% \setcounter{tocdepth}{3}

% %Inkludiert das Literaturverzeichnis ins Inhaltsverzeichnis
% %\usepackage[nottoc]{tocbibind}
% %Referenz: ftp://ftp.tex.ac.uk/tex-archive/macros/latex/contrib/tocbibind/tocbibind.pdf

% % Für sehr, sehr viel besseren Blocksatz
% \usepackage[activate={true,nocompatibility},final,tracking=true,kerning=true,spacing=true,factor=1100,stretch=10,shrink=10]{microtype}
% % activate={true,nocompatibility} - activate protrusion and expansion
% % final - enable microtype; use "draft" to disable
% % tracking=true, kerning=true, spacing=true - activate these techniques
% % factor=1100 - add 10% to the protrusion amount (default is 1000)
% % stretch=10, shrink=10 - reduce stretchability/shrinkability (default is 20/20)

% % Macht Leerzeilen zu 'richtigen' Absätzen
% \usepackage{parskip}

% %Einzug der Fußnoten
% %\usepackage[hang, bottom]{footmisc}
% %\setlength{\footnotemargin}{15pt}

% %Fußnoten von Grafiken auf derselben Seite
% \usepackage{afterpage}

% % URL-Paket
% \usepackage[colorlinks,        % Links ohne Umrandungen in zu wählender Farbe
% 	linkcolor=black,   % Farbe interner Verweise
% 	filecolor=black,   % Farbe externer Verweise
% 	citecolor=black,
% 	urlcolor=black]{hyperref} 
	
% %Erstellung eines Glossars
% %\usepackage[nomain, acronym, nonumberlist, toc, section=section, nopostdot]{glossaries}
% %\renewcommand*{\glsgroupskip}{}% no extra line skip between clusters
% %\makeglossaries


% %\input{Glossar.tex}
% %\input{Abkürzungen.tex}

% %Bessere Beschreibungslisten
% %\usepackage{enumitem}

% %Erstellung von Tabellen
% \usepackage[margin=00pt,font=small,labelfont=bf, position=below, skip=3pt, justification=justified]{caption}
% \usepackage{booktabs} % Schickere Linien in Tabellen
% \usepackage{colortbl} % Farben in Tabellen
% \usepackage{tabularx}
% %\usepackage{multirow}

% % Kapitelweise Nummerierung von Abbildungen
% %\usepackage{chngcntr}
% %\counterwithin{figure}{section}

% %Einbinden von externen PDF-Dateien
% \usepackage{pdfpages}

% % Überschriftenformatierung ermöglichen
% %\usepackage{titlesec}

% %Subparagraphen kursiv
% \titleformat*{\subparagraph}{\itshape}

% %Blindtext
% \usepackage{blindtext}

% %Mathematische Symbole
% \usepackage[geometry]{ifsym}

% %Farbiger Text
% \usepackage{color}
% \definecolor{grau}{gray}{.6}

% %Fußnoten ohne Ziffer
% \makeatletter
% \def\blfootnote{\gdef\@thefnmark{}\@footnotetext}
% \makeatother

% % EINSTELLUNGEN
% % -------------------------------------------

% % Verhindert Hurenkinder und Schusterjungen recht zuverlässig
% \widowpenalty = 10000
% \clubpenalty = 10000
% \interfootnotelinepenalty = 10000

% %Einzug in Beschreibungslisten global festlegen
% \setlist[description]{leftmargin=\einzug}

% %Abstand zwischen Gleitobjekten und Text
% \setlength{\intextsep}{20pt}