\section{Entropy}

The strength of a password can be described by the number of guesses attackers need to crack it. The more tries an hacker needs to find the password, the more secure it is \cite{passwordquality}. Length and element complexity of the password play into account how large this number will be \cite{passwordstrength}. This results in two problems when trying to measure the security of a password.

For one this number is highly subjective as every attacker would need a different number of guesses. It's difficult to judge how many tries the average hacker might need. Also, this number can only be stated after an attack was successful. Instead the entropy can be used to objectively predict this event and measure the strength of randomly chosen passwords.

\begin{equation}
H = log_2(N^L)
\end{equation}

The entropy H of a password can be determined based on its length L and the pool size N each element can be chosen from. Increasing the possible complexity of each element or increasing the length will also increase the entropy of a password \cite{webernetz}.

\begin{table}[h!]
\centering
\begin{tabular}{l l}
    Length of Password (L)  & Entropy \\
    \hline
    6                       & 28 \\
    8                       & 38 \\
	10                      & 47 \\
    12                      & 56 \\
    14						& 66
\end{tabular}
\caption{Entropy based on password length. N = 26}
\end{table}

Because of its exponential growth increasing the length of a password will have a larger effect on the entropy than a more complex pool size \cite{webernetz}. Therefore users who are interested in strengthening their passwords should consider making it longer rather than using additional symbol groups like special characters. Increasing both will still get the best results.

\begin{table}[h!]
\centering
\begin{tabular}{l l}
    Character Complexity (N) & Entropy \\
    \hline
    10                       & 27 \\
    26                       & 38 \\
    36                       & 41 \\
    52                       & 46 \\
    95                       & 53
\end{tabular}
\caption{Entropy based on a character complexity. L = 8}
\end{table}

Passwords that are made up of actual words are called passphrases. Calculating the entropy of ``correcthorsebatterystaple'' using the entropy equation based on single characters results in 118. However instead of analysing this passphrase based on characters, one could also do it based on the four words it consists of. ``correct'', ``horse'', ``battery'' and ``staple'' are all in the top 2000 of the most common english words \cite{commonwords}. Instead of seeing a tuple made of 25 lowercase characters, it is also possible to see a tuple with the length of 4 and a pool size of 2000. The same entropy equation is used, but since this is based on a different model the result changes --- in this case to 44.

\begin{equation}
H_{chars} = log_2(26^{25}) = 118
\end{equation}
\begin{equation}
H_{words} = log_2(2000^4) = 44
\end{equation}

The actual entropy of this password depends on the method that was used to generate it. If ``correcthorsebatterystaple'' is the result of 25 randomly chosen lowercase characters than its entropy would be 118. If instead it was created by using 4 random words from the 2000 most common words than the entropy would be 44. However, since we should predict the strength of a password from the perspective of an attacker, the entropy cannot be higher than the lowest rating we can find to calculate it.

Another problem emerges when taking into account passwords that were chosen by humans. These passwords cannot be seen as truly random, because every person will include some level of self revelation. In theory it is still possible to use the entropy equation to calculate its strength. In practice it is almost impossible to choose N and L in a way that accounts for the subjectivity of the person.

One approach to this gap between the entropy of a random password and the realistic strength is used by a software called zxcvbn \cite{zxcvbn}. It tries to account for patterns and predictable information users might include in their passwords. For example "password" has an entropy of 38 calculated on a character basis and an entropy of 9 when analysed as a word. However zxcvbn rates its entropy as 0, because its known as the most common password in the world.

\newpage